\documentclass[onecolumn, draftclsnofoot,10pt, compsoc]{IEEEtran}
\usepackage{graphicx}
\usepackage{url}
\usepackage{setspace}
\usepackage{listings}

\usepackage{geometry}
\geometry{textheight=9.5in, textwidth=7in}

% 1. Fill in these details
\def \CapstoneTeamName{		Skill Capped IRL}
\def \CapstoneTeamNumber{		17}
\def \GroupMemberOne{			Katherine Bajno}
\def \GroupMemberTwo{			Meagan Olsen}
\def \GroupMemberThree{			William Sims}
\def \GroupMemberFour{			Kiarash Teymoury}
\def \CapstoneProjectName{		eBay iOS eSports Application}
\def \CapstoneSponsorCompany{	eBay}
\def \CapstoneSponsorPerson{		Luther Boorn}

% 2. Uncomment the appropriate line below so that the document type works
\def \DocType{		%Problem Statement
				%Requirements Document
				%Technology Review
				%Design Document
				Progress Report
				}
			
\newcommand{\NameSigPair}[1]{\par
\makebox[2.75in][r]{#1} \hfil 	\makebox[3.25in]{\makebox[2.25in]{\hrulefill} \hfill		\makebox[.75in]{\hrulefill}}
\par\vspace{-12pt} \textit{\tiny\noindent
\makebox[2.75in]{} \hfil		\makebox[3.25in]{\makebox[2.25in][r]{Signature} \hfill	\makebox[.75in][r]{Date}}}}
% 3. If the document is not to be signed, uncomment the RENEWcommand below
%\renewcommand{\NameSigPair}[1]{#1}

%%%%%%%%%%%%%%%%%%%%%%%%%%%%%%%%%%%%%%%
\begin{document}
\begin{titlepage}
    \pagenumbering{gobble}
    \begin{singlespace}
    	% Need to uncomment this line below to include the COE photo
    	%\includegraphics[height=4cm]{coe_v_spot1}
        \hfill 
        % 4. If you have a logo, use this includegraphics command to put it on the coversheet.
        %\includegraphics[height=4cm]{CompanyLogo}   
        \par\vspace{.2in}
        \centering
        \scshape{
            \huge CS Capstone \DocType \par
            {\large\today}\par   
            \vspace{.5in}
            \textbf{\Huge\CapstoneProjectName}\par
            {\large CS461 Senior Software Engineering Project I}\par
            {\large Fall 2017}\par
            \vfill
            {\large Prepared for}\par
            \Huge \CapstoneSponsorCompany\par
            \vspace{5pt}
            {\Large\NameSigPair{\CapstoneSponsorPerson}\par}
            {\large Prepared by }\par
            Group\CapstoneTeamNumber\par
            % 5. comment out the line below this one if you do not wish to name your team
            \CapstoneTeamName\par 
            \vspace{5pt}
            {\Large
                \NameSigPair{\GroupMemberOne}\par
                \NameSigPair{\GroupMemberTwo}\par
                \NameSigPair{\GroupMemberThree}\par
                 \NameSigPair{\GroupMemberFour}\par
            }
            \vspace{20pt}
        }
        \begin{abstract}
        % 6. Fill in your abstract
        This document outlines the progress made on the eBay iOS eSports application during fall term.
        Included is a discussion of project goals, week by week summaries, and an overall retrospective of the past ten weeks.
            
        	%This document is written using one sentence per line.
        	%This allows you to have sensible diffs when you use \LaTeX with version control, as well as giving a quick visual test to see if sentences are too short/long.
        	%If you have questions, ``The Not So Short Guide to LaTeX'' is a great resource (\url{https://tobi.oetiker.ch/lshort/lshort.pdf})
        \end{abstract}     
    \end{singlespace}
\end{titlepage}
\newpage
\pagenumbering{arabic}
\tableofcontents
% 7. uncomment this (if applicable). Consider adding a page break.
%\listoffigures
%\listoftables
\clearpage

% 8. now you write!

\section{Purpose and Goals}

\subsection{Purpose}
eBay would like to make use of recently released public APIs in order to explore the eSports market and target new customers. 
Currently, there aren’t any known products that make it easy for eSports fans to find merchandise and receive updates about their favorite games. 
The purpose of our application is to showcase the new eBay APIs and attract millennial gamers. 

\subsection{Goals}
The goals for this term were to develop a technical roadmap for the rest of the year in order to satisfy the needs of our client. 
We defined the problem we are trying to solve, outlined requirements, researched different technologies, and created a design plan. 
We also created a user interface design mock-up to inform the visual aspect of our project as we begin development next term. 

\section{Weekly Progress}

\subsection{Week 1}

\subsubsection{Activities}
Since this was just our first week, we didn’t have many things we were assigned to do yet. Much of this first week was spent learning the process flows and rules of this three-part senior capstone class. The potential projects came up early in the week, and since our team was not formed yet, we each reviewed and researched the projects extensively. Some of the members on the team reached out to the eBay client, Luther Boorn, requesting to be put on this project. In addition, we each created our own OneNote pages with an introduction and summary about ourselves and shared it with our instructors. By the end of the week, we were required to submit our project preferences and justifications for them. 

\subsection{Week 2}

\subsubsection{Activities}
This week project assignments were posted. After class that day, we met up with our team members and exchanged contact information. We set up a weekly meeting time with the team and our TA. We also reached out to our client and setup a Slack channel to communicate with him. In class, we were assigned a problem statement, detailing our project from a ten-thousand foot overview. Each team member was required to create this individually, and then we were to meet up and combine the individual problems statements into one unified rough draft.

\subsubsection{Problems and Solutions}
We ran into a roadblock because we didn’t get to meet with our client for the first time until later in the week. This was problematic because without meeting with our client, it was difficult to compose a rough draft of the problem statement. We solved this problem by requesting an extension from our instructor, which gave us until the following Monday to complete the rough draft. Once we had our first remote meeting with the client, composing the problem statements was trivial.

\subsection{Week 3}

\subsubsection{Activities}
This week we set up a repository using GitHub for the project and shared it with our instructors. We also had our first meeting with the TA and client. In our team meeting, we worked to combine our problem statements into a final draft. We also discussed the problem statement with our client before submitting it. Our problem statement was successfully uploaded to GitHub by the due date. However, one of the team members had difficulties getting the file into the compressed tar.bz2 format that was requested by the instructor.

\subsubsection{Problems and Solutions}
One of the biggest problems we ran into this week was that not everyone in the group had their own Macintosh laptop. Since our application is being built in iOS, it is critical that all group members have access to OSX and an iPhone for testing. This problem was not fixed right away, but by the end of the term all group members had access to a mac computer running Xcode 9.

\subsection{Week 4}
\subsubsection{Activities}
This week, we sent our final problem statement to our client, Luther, for a signature approval. Some of the team members started looking into Xcode and Swift tutorials. As a group, we set up a time to meet with our client in Portland, OR, where their local office is. This week, we also were able to smoothly transition our TA meetings to be remote. This was advantageous since we all had busy schedules in the afternoon and one of our teammates is remote anyways.
\subsubsection{Problems and Solutions}
We still had the problem with the Macs at this point. This was a huge hindrance, because it prevented some of us from getting set up with the Xcode development environment. 

\subsection{Week 5}
\subsubsection{Activities}
This week, we were assigned a requirements document to start working on. This was to be a team document, and our rough draft was due by the end of the week. We also were able to meet with our client in person for the first team. He spent a couple hours with us showing us the office, laying out the high-level requirements for our project, and drafting out some user interface sketches for us.
\subsubsection{Problems and Solutions}
A major problem we ran into this week was the fact that the rough draft of the requirements document was due at midnight while we hadn’t met with our client since the first meeting. This made it difficult to start drafting the requirements document, but the meeting with the client helped provide us with the clarity we needed to get a decent rough draft. The only downside was that we had to rewrite some portions of the requirements document following our meeting with the client.
\subsection{Week 6}
\subsubsection{Activities}
This week was mainly concerned with finishing up the requirements document. However, we were also able to start creating some hi-fi user interface designs in Sketch based off of the meeting with our client the week before. With the help of our client, we were also able to get an extension from our instructor for our final version of the requirements document. We were in communication with the TA about this extension. The members that had their macs started getting Xcode installed on their machines.

\subsubsection{Problems and Solutions}
The largest problem we had was the due date for our requirements document. The reason that we requested an extension was to get our client’s approval before the due date. This problem was easily solved by communicating with the instructor about our situation. One of the team members also had some difficulty setting up Xcode on their machine.
\subsection{Week 7}
\subsubsection{Activities}
This week we were each assigned a tech review for a part of our project. This was an individual assignment, so we had to divide up the pieces of our project. Some of the team members started reading Swift documentation on the Apple website. We also divided up the user interface screens to design. Besides some minor updates to Xcode, we also got to meet four mentors provided by our client on the slack channel. We also uploaded the requirements document to GitHub and OneNote after obtaining a signature from our client.

\subsubsection{Problems and Solutions}
Our biggest obstacle this week was the discovery that one of our group member’s Macintosh OS was not compatible with the version of Xcode that we need for the project. This was a huge set back because the OS could not be installed on the existing hardware of the mac. Unfortunately, that put us into another problem trying to make sure all of our group members had necessary development tools. We also had a hard time as a team coming up with enough pieces for all four of our group members. We resolved this by talking with our client, who had gone through this last year with his team and also by speaking with our instructor, Kevin. We also did some independent research.

\subsection{Week 8}
\subsubsection{Activities}
This week was mainly concerned with completing our rough drafts of the tech reviews. Now that we had all of our technical resources available to us via Slack, we decided to start setting up weekly remote meetings with our client and mentors for the interim. We continued to learn Swift, and we were able to meet with our client and get feedback for our final drafts.

\subsubsection{Problems and Solutions}
Our biggest struggle this week was coming up with enough pieces for all four of our group members. We resolved this by talking with our client, who had gone through this last year with his team and also by speaking with our instructor, Kevin. We also conducted independent research.

\subsection{Week 9}
\subsubsection{Activities}
This week was concerned with finishing our final drafts for the technology reviews, figuring out how to cite sources and fixing up our rough drafts from the peer review feedback in class. During this week, we  were assigned two large documents to complete by finals week: the design document and an end of the term progress report and presentation. We started planning the large design document. Some of the group members also looked into different methods for UI development. Our client also posted a dummy application using xibs for us to play around with. 

\subsubsection{Problems and Solutions}
Our only obstacle this week was the holiday. Because of Thanksgiving, we didn’t get to meet with our TA which caused a bit of a setback on our design document. However, we were able to communicate and work through our misunderstandings about the design document. 

\subsection{Week 10}
\subsubsection{Activities}
This week, we all finished our final drafts of the design documents and obtained signatures from the client. We finished up the design document and started planning for the presentation. We also started on the progress report. We set up a meeting for the following week to discuss the design document with our client. We also received feedback on our technology reviews from our technical mentors.

\section{Retrospective}
\begin{center}
    \begin{tabular}{ | p{5cm} | p{5cm} | p{5cm} |}
    \hline
    \textbf{Positives} & \textbf{Deltas} & \textbf{Actions} \\ \hline
   Communication with client and with each other & Dividing up work sooner & Make sure we do enough planning before jumping straight into individual parts \\ \hline
     Proactiveness and organization about everything & Starting work early in case people run into issues & Make sure to be courteous of other team members by making a priority to get group-related work done as early as possible before the due date  \\ \hline
     Enthusiasm about project  & Listen to other teammates when they make suggestions & Figure out work assignments earlier in the process \\ \hline
      Prepared well for meetings & Meeting over video conference more often so communication is more clear & Be more to communication when conflicts arise\\ \hline
       Inclusiveness and involvement of everyone in group &   & Use video conferencing more often to solve problems \\ \hline
     Being understanding of each other’s busy schedules  &    & \\ \hline
      Helping each other out in jams  & &\\ \hline
    \end{tabular}
\end{center}

\section{Current State of the Project}

Now that we have reached the end of the term, we are confident about where the application is standing. We fully understand the problem we are solving and have looked at all the different tools and possible ways to implement the different aspects of the application. In addition, every team member is fully aware of their responsibilities of each software component needed. Lastly, we have a significant portion of the UI mock-ups done which we are hoping to complete as we progress toward next term. The team has decided to start development over the winter break to get more familiar with Apple's SDK and different functionalities of Xcode.

\end{document}