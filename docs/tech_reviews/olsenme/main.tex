\documentclass[onecolumn, draftclsnofoot,10pt, compsoc]{IEEEtran}
\usepackage{graphicx}
\usepackage{url}
\usepackage{setspace}
\usepackage{pgfgantt}

\newcommand{\subparagraph}{}
\usepackage{titlesec}
\setcounter{secnumdepth}{4}

\usepackage{geometry}
\geometry{textheight=9.5in, textwidth=7in}

% 1. Fill in these details
\def \CapstoneTeamName{		Skill Capped IRL}
\def \CapstoneTeamNumber{		17}
\def \GroupMemberOne{			Katherine Bajno}
\def \GroupMemberTwo{			Meagan Olsen}
\def \GroupMemberThree{			William Sims}
\def \GroupMemberFour{			Kiarash Teymoury}
\def \CapstoneProjectName{		eBay iOS eSports Application}
\def \CapstoneSponsorCompany{	eBay}
\def \CapstoneSponsorPerson{		Luther Boorn}

% 2. Uncomment the appropriate line below so that the document type works
\def \DocType{	%Problem Statement
				%Requirements Document
				Technology Review
				%Design Document
				%Progress Report
				}
			
\newcommand{\NameSigPair}[1]{\par
\makebox[2.75in][r]{#1} \hfil 	\makebox[3.25in]{\makebox[2.25in]{\hrulefill} \hfill		\makebox[.75in]{\hrulefill}}
\par\vspace{-12pt} \textit{\tiny\noindent
\makebox[2.75in]{} \hfil		\makebox[3.25in]{\makebox[2.25in][r]{Signature} \hfill	\makebox[.75in][r]{Date}}}}
% 3. If the document is not to be signed, uncomment the RENEWcommand below
%\renewcommand{\NameSigPair}[1]{#1}

%%%%%%%%%%%%%%%%%%%%%%%%%%%%%%%%%%%%%%%
\begin{document}
\begin{titlepage}
    \pagenumbering{gobble}
    \begin{singlespace}
    	% Need to uncomment this line below to include the COE photo
    	%\includegraphics[height=4cm]{coe_v_spot1}
        \hfill 
        % 4. If you have a logo, use this includegraphics command to put it on the coversheet.
        %\includegraphics[height=4cm]{CompanyLogo}   
        \par\vspace{.2in}
        \centering
        \scshape{
            \huge CS Capstone \DocType \par
            {\large\today}\par   
            \vspace{.5in}
            \textbf{\Huge\CapstoneProjectName}\par
            {\large CS461 Senior Software Engineering Project I}\par
            {\large Fall 2017}\par
            \vfill
            {\large Prepared for}\par
            \Huge \CapstoneSponsorCompany\par
            \vspace{5pt}
            {\Large\NameSigPair{\CapstoneSponsorPerson}\par}
            {\large Prepared by }\par
            Group\CapstoneTeamNumber\par
            % 5. comment out the line below this one if you do not wish to name your team
            \CapstoneTeamName\par 
            \vspace{5pt}
            {\Large
                \NameSigPair{\GroupMemberOne}\par
                \NameSigPair{\GroupMemberTwo}\par
                \NameSigPair{\GroupMemberThree}\par
                 \NameSigPair{\GroupMemberFour}\par
            }
            \vspace{20pt}
        }
         \begin{abstract}
        % 6. Fill in your abstract
          This project is to create a mobile application which runs on an iOS platform.This application will display information about upcoming eSports events as well as available eBay eSports merchandise related to these events.In this project, we will be using Swift as a development framework.Navigation through the application will happen via tab bars.
        	%This document is written using one sentence per line.
        	%This allows you to have sensible diffs when you use \LaTeX with version control, as well as giving a quick visual test to see if sentences are too short/long.
        	%If you have questions, ``The Not So Short Guide to LaTeX'' is a great resource (\url{https://tobi.oetiker.ch/lshort/lshort.pdf})
        \end{abstract}  
    \end{singlespace}
\end{titlepage}
\newpage
\pagenumbering{arabic}
\tableofcontents
% 7. uncomment this (if applicable). Consider adding a page break.
%\listoffigures
%\listoftables
\clearpage
\section{Piece 1:Platform}
	
\subsection Android 
	Android is an open source Linux based platform which is used to build applications which run on many different types of devices including phones, televisions, and tablets.Being Linux based allows the platform to leverage some of the added features of the Linux operating system and  creates a common ground for the creation of new hardware devices.For example, the Android Runtime(ART) is able to rely on the Linux Kernel for implementation of features such as threading and memory management.Android's implementation of the Linux user based model also ensures that each application runs as its own process with restricted permissions.Furthermore, each process also has its own virtual machine to ensure isolation from other applications. Each application also has its own Android Runtime(ART), which optimizes memory usage through optimization of bytecode.Some distinct features of the ART are its support for Ahead of Time(AOT) and Just in Time(JIT) compilation, its optimized implementation of garbage collection, and added debugging support.Android also provides support for the Java Runtime Libraries, allowing for a convenient, easy to use,Java development environment.Android also allows developers to take advantage of native c/c++ libraries through a Java API. The Java API framework also allows for code reuse ad developers are able to have access to all the framework API that Android itself uses.Android also provides some built in applications that developers can use such as applications for email, SMS messaging, calendars, internet browsing,and contacts.Another advantage of Android is its device configuration support built into the applications themselves. This  allows for an application to be configured to run on devices of all different screen sizes.
 \subsection Windows 
   Unlike the Android development platform which is based on Linux, Universal Windows platform is based on the Windows 10 Operating system. Like Android, Universal Windows Platform provides a common interface for development on multiple different devices including  PC, tablet, phone, Xbox, HoloLens, Surface Hub and more. Like Android, it also provides support for development on a wide range of screen sizes through a variety of layout panels. One advantage of the Universal Windows Platform is its support for languages such as C sharp and XAML.A major advantage of the Universal Windows platform is its device support across all devices running Windows.Windows also provides customizable software development kits which allow applications to be configured to evoke special features on devices that have support for them.In addition, all Windows Applications are packaged allowing for an easy installation.Windows applications allow developers to provide real time data to help keep the application relevant and exciting.Windows Universal platform provides a scaling mechanism for UI elements to keep them a consistent size across all platforms. 
 \subsection IOS 
 	The iOS platform is used to specifically run iPad, iPhone, and iPod touch devices.Similar to the Android platform, the iOS platform also provides built in applications which accompany the operating system. Similar to Windows packaging, iOS comes with a development kit which allows for testing, running, and installing native applications.iOS also packages their resources and shared libraries into frameworks that can be imported through the xCode IDE.  Unlike Android, which is Java based, iOS applications are built on the Objective C framework and run entirely on iOS. Similar to the structure of Android,iOS has a layered architecture, where the OS core features are at the bottom and the higher level services are found at the top.At the Operating systems level, iOS is based on Mach and uses unix interfaces.Similar to Android, iOS also provides its own security framework. Inside this framework are APIS for performing various types of encryption and pseudorandom number generation.It also has the added support of a robust Crypto library which allows for symmetric encryption, hash-based message authentication codes (HMACs), and digests.The reason that we chose this platform is because our goal was to develop for an iOS device.Because it is based on the objective C framework, it served as a good introduction to mobile application development given our backgrounds in C.The corporation that we are developing for also has a large development department that develops specifically for this platform.
\section{Piece 2: Framework}    
  \subsection Objective C
   Objective C is the development framework for iOS.In the past, it served as the primary programming language for writing OSX and IOS software.It was initially built as an extension to the ANSI C programming language for the purpose of allowing for complex object oriented programming without the complex syntax of C++. Its simple syntax makes it easy to learn while still not restricting the power of an object oriented language. It includes dynamic runtime capabilities, making it a flexible language for design as users have much more freedom to specify things at runtime. Because of this, this framework is known to be a good choice for development and design of front end GUIS.One advantage of this technology is that all of Apple's documentation is based off of it, so it has a wide range of support and documentation. 
   
  \subsection Xarmin 
   Xarmin is a tool used for mobile development. In contrast to Objective C, Xarmin allows for both native and cross platform development allowing developers to create for Android, iOS and Windows with a 75 percent code sharing capability through its C sharp interfaces.In contrast to Objective C, which is made for the Xcode development environment, Xarmin actually has its own development environment.It also has support to interact with Microsoft's Visual Studio development environment.Xarmin makes use of the c sharp programming language, which is a benefit for those accustomed to Windows development.It also allows for integration with Microsoft's Mono.Net framework.For iOS, Xarmin supports Ahead of Time Compiling(AOT) to ARM assembly language.One advantage of Xarmin is that it has built in tools for platform specific hardware acceleration which cannot be achieved in frameworks like objective C.Another advantage that Xarmin has over objective C and swift is its support for asynchronous programming, which requires callbacks in objective C and Swift.It also simplifies lambdas, which are more difficult to implement in Objective C.Another syntactic advantage that Xarmin has over objective C is that it actually supports objective C code, frameworks, and custom controls.An advantage that this framework could have for Android development is that it performs just in time(JIT) compilation to optimize performance.In the same way that it allows for Objective C code to be called for iOS,Xarmin also has support for Java code from Android.Another thing that makes this platform extremely versatile is that it allows access to iOS, Windows, and Android software development kits.It also allows developers to use design tools for Windows, iOS, and Android for UI design.
   \subsection Swift 
   Swift is the newer, more modern, evolved version of Objective C and is now starting to replace it in many applications for all OsX devices.This framework can be used for desktops, servers, mobile devices, and many other things.In addition to Apple contributions, Swift also incorporates modifications from the open source community.Similar to Xarmin, swift is optimized for hardware.However unlike Xarmin, swift does not contain cross platform capability.In contrast to Objective C, Swift has lightweight syntax that is optimized for developers while still allowing for complex logic.It has a playground which allows for easy testing and exploration of code.An advantage that is has for previous iOS developers is that it is syntactically similar to Objective C syntax.It also has advanced types such as tuples which are not supported in Objective C.It also replaces nil pointers in Objective C, which makes it much safer and less error prone.It also has an added type interface that is not found in Objective C. This helps prevent type errors and also favors fewer type declarations.Swift is compatible with all current iOS tools.Another advantage is that it allows integration with objective C and also enhances objective c with the features of cocoa touch.
\Section{Piece 3:Navigation}
  \subection Hamburger Menu
   The hamburger menu is a commonly used navigational method for both mobile and web applications. It is also commonly called a side navigation drawer. It is usually implemented by a carousel with bars somewhere near the top of a page which, when clicked, pulls out a menu of some kind. While these were initially not part of the apple API, they have become a popular design pattern on Android devices and are implemented natively on that platform.One advantage to this style of navigation is that it allows more space for emphasis of the main content. This can be particularly advantageous on a mobile platform where space is limited. Another advantage that this navigation style ads is the ability to display multiple navigational links that might clutter up the screen on a mobile device.
 \subsection Tab Bars
  	Tab bars are a design pattern that show up frequently in both Android and iOS mobile applications. In iOS, they are more commonly known as a navigation bar whereas Android refers to them as toolbars.In iOS, this navigation style usually contains the title of the current page that the user is on, some sort of arrow to navigate to a previous page, and some sort of filter to control the content of the current page.This design choice is useful in situations where there are fewer navigation choices. It is often implemented when the application uses a hierarchical structure. Our application does implement this design choice as a part of many of our interfaces.Since our application does not present many navigation options, this provides a sound,easy to use navigational style.While our client did suggest that we use this, he also has given us the freedom to customize UI elements such as these.
 \subsection Tabs
    Tabs are a design pattern that also shows up quite frequently in both iOS and Android. In iOS, tabs usually appear at the bottom of the screen whereas in Android, they are usually found towards the top unless the primary navigation shows up at the bottom of the application.This design pattern is often a good choice for simplicity. Taken from the desktop model, it provides direct access to many links and leaves no ambiguity about what page the end user is on.This is often useful with an application that performs several distinct things because it provides the user a clear indication of where they are on an application. Another advantage that these tabs have is persistence. They always stay on the page.Unlike the hamburger menu which has the capacity to hold many links, tabs are limited in the amount of elements that they can display. Having too many tabs clutters up the screen too much. We chose this design pattern as our primary navigation mainly because we have two main functionalities:display products and display events. These two pieces are distinct actions, so this seemed like a solid design choice. We also have a home page for a total of three tabs.
    
   
  
\end{document}