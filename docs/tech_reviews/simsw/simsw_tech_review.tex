\documentclass[onecolumn, draftclsnofoot,10pt, compsoc]{IEEEtran}
\usepackage{graphicx}
\usepackage{url}
\usepackage{setspace}
\usepackage{pgfgantt}

\newcommand{\subparagraph}{}
\usepackage{titlesec}
\setcounter{secnumdepth}{4}

\usepackage{geometry}
\geometry{textheight=9.5in, textwidth=7in}

% 1. Fill in these details
\def \CapstoneTeamName{		Skill Capped IRL}
\def \CapstoneTeamNumber{		17}
%def \GroupMemberOne{			Katherine Bajno}
%\def \GroupMemberTwo{			Meagan Olsen}
\def \GroupMemberThree{			Will Sims}
%\def \GroupMemberFour{			Kiarash Teymoury}
\def \CapstoneProjectName{		eBay iOS eSports Application}
\def \CapstoneSponsorCompany{	eBay}
\def \CapstoneSponsorPerson{		Luther Boorn}

% 2. Uncomment the appropriate line below so that the document type works
\def \DocType{	%Problem Statement
				%Requirements Document
				Technology Review
				%Design Document
				%Progress Report
				}
			
\newcommand{\NameSigPair}[1]{\par
\makebox[2.75in][r]{#1} \hfil 	\makebox[3.25in]{\makebox[2.25in]{\hrulefill} \hfill		\makebox[.75in]{\hrulefill}}
\par\vspace{-12pt} \textit{\tiny\noindent
\makebox[2.75in]{} \hfil		\makebox[3.25in]{\makebox[2.25in][r]{Signature} \hfill	\makebox[.75in][r]{Date}}}}
% 3. If the document is not to be signed, uncomment the RENEWcommand below
%\renewcommand{\NameSigPair}[1]{#1}

%%%%%%%%%%%%%%%%%%%%%%%%%%%%%%%%%%%%%%%
\begin{document}
\begin{titlepage}
    \pagenumbering{gobble}
    \begin{singlespace}
    	% Need to uncomment this line below to include the COE photo
    	%\includegraphics[height=4cm]{coe_v_spot1}
        \hfill 
        % 4. If you have a logo, use this includegraphics command to put it on the coversheet.
        %\includegraphics[height=4cm]{CompanyLogo}   
        \par\vspace{.2in}
        \centering
        \scshape{
            \huge CS Capstone \DocType \par
            {\large\today}\par   
            \vspace{.5in}
            \textbf{\Huge\CapstoneProjectName}\par
            {\large CS461 Senior Software Engineering Project I}\par
            {\large Fall 2017}\par
            \vfill
            {\large Prepared for}\par
            \Huge \CapstoneSponsorCompany\par
            \vspace{5pt}
            {\Large\NameSigPair{\CapstoneSponsorPerson}\par}
            {\large Prepared by }\par
            Group\CapstoneTeamNumber\par
            % 5. comment out the line below this one if you do not wish to name your team
            \CapstoneTeamName\par 
            \vspace{5pt}
            {\Large
                %\NameSigPair{\GroupMemberOne}\par
                %\NameSigPair{\GroupMemberTwo}\par
                \NameSigPair{\GroupMemberThree}\par
                % \NameSigPair{\GroupMemberFour}\par
            }
            \vspace{20pt}
        }
        \begin{abstract}
        % 6. Fill in your abstract
        This document contains a review of three different pieces of technology that will be neccesary to complete the objectives outlined in our requirements document. 
        I will be responsible for reviewing different methods of generating Twitter data, generating Event data, and converting JSON strings into Swift data types. 
        Our project involves the creation of an iOS application by utilizing eBay public buying APIs to help eBay understand the eSports market and provide shopping opportunities. 
        My role is focus on the social component where we will display tweets from targeted eSports accounts in order to make the application more engaging.
            
        	%This document is written using one sentence per line.
        	%This allows you to have sensible diffs when you use \LaTeX with version control, as well as giving a quick visual test to see if sentences are too short/long.
        	%If you have questions, ``The Not So Short Guide to LaTeX'' is a great resource (\url{https://tobi.oetiker.ch/lshort/lshort.pdf})
        \end{abstract}     
    \end{singlespace}
\end{titlepage}
\newpage
\pagenumbering{arabic}
\tableofcontents
% 7. uncomment this (if applicable). Consider adding a page break.
%\listoffigures
%\listoftables
\clearpage

% 8. now you write!
\section{Introduction}
\section{Piece 1: Generating Twitter Data}
\subsection{Overview}
\subsection{Potential Choices}
\subsubsection{Twitter Kit for iOS\cite{twitter}}
\subsubsection{Scraping HTML Data with SwiftSoup\cite{ssoup}}
\subsubsection{Manually adding tweets to Firebase\cite{fb}}
\subsection{Discussion}
\subsection{Conclusion}

\section{Piece 2: Generating Event Data}
\subsection{Overview}
\subsection{Potential Choices}
\subsubsection{Manually adding events to Firebase\cite{fb}}
\subsubsection{Toornament API\cite{toornament}}
\subsubsection{Abios API\cite{abios}}
\subsection{Discussion}
\subsection{Conclusion}

\section{Piece 3: Converting JSON}
\subsection{Overview}
\subsection{Potential Choices}
\subsubsection{Apple Foundation Framework\cite{json}}
\subsubsection{SwiftyJSON Library\cite{sjson}}
\subsubsection{Gloss Library\cite{gloss}}
\subsection{Discussion}
\subsection{Conclusion}

\bibliography{tech_review}
\bibliographystyle{IEEEtran}

\end{document}