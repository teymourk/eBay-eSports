\documentclass[onecolumn, draftclsnofoot,10pt, compsoc]{IEEEtran}
\usepackage{graphicx}
\usepackage{url}
\usepackage{setspace}
\usepackage{pgfgantt}

\newcommand{\subparagraph}{}
\usepackage{titlesec}
\setcounter{secnumdepth}{4}

\usepackage{geometry}
\geometry{textheight=9.5in, textwidth=7in}

\usepackage{listings}
% Taken from http://timmurphy.org/2014/01/27/displaying-code-in-latex-documents/ with some modifications
\lstset{
    breaklines=true
    frame=none,
    tabsize=4, % tab space width
    showstringspaces=false, % don't mark spaces in strings
    numbers=none, 
    commentstyle=\color{green}, % comment color
    keywordstyle=\color{blue}, % keyword color
    stringstyle=\color{red}, % string color
    basicstyle={\small\ttfamily}, % font style
    belowcaptionskip=1\baselineskip
}

% 1. Fill in these details
\def \CapstoneTeamName{		Skill Capped IRL}
\def \CapstoneTeamNumber{		17}
%def \GroupMemberOne{			Katherine Bajno}
%\def \GroupMemberTwo{			Meagan Olsen}
\def \GroupMemberThree{			Will Sims}
%\def \GroupMemberFour{			Kiarash Teymoury}
\def \CapstoneProjectName{		eBay iOS eSports Application}
\def \CapstoneSponsorCompany{	eBay}
\def \CapstoneSponsorPerson{		Luther Boorn}

% 2. Uncomment the appropriate line below so that the document type works
\def \DocType{	%Problem Statement
				%Requirements Document
				Technology Review
				%Design Document
				%Progress Report
				}
			
\newcommand{\NameSigPair}[1]{\par
\makebox[2.75in][r]{#1} \hfil 	\makebox[3.25in]{\makebox[2.25in]{\hrulefill} \hfill		\makebox[.75in]{\hrulefill}}
\par\vspace{-12pt} \textit{\tiny\noindent
\makebox[2.75in]{} \hfil		\makebox[3.25in]{\makebox[2.25in][r]{Signature} \hfill	\makebox[.75in][r]{Date}}}}
% 3. If the document is not to be signed, uncomment the RENEWcommand below
%\renewcommand{\NameSigPair}[1]{#1}

%%%%%%%%%%%%%%%%%%%%%%%%%%%%%%%%%%%%%%%
\begin{document}
\begin{titlepage}
    \pagenumbering{gobble}
    \begin{singlespace}
    	% Need to uncomment this line below to include the COE photo
    	%\includegraphics[height=4cm]{coe_v_spot1}
        \hfill 
        % 4. If you have a logo, use this includegraphics command to put it on the coversheet.
        %\includegraphics[height=4cm]{CompanyLogo}   
        \par\vspace{.2in}
        \centering
        \scshape{
            \huge CS Capstone \DocType \par
            {\large\today}\par   
            \vspace{.5in}
            \textbf{\Huge\CapstoneProjectName}\par
            {\large CS461 Senior Software Engineering Project I}\par
            {\large Fall 2017}\par
            \vfill
            {\large Prepared for}\par
            \Huge \CapstoneSponsorCompany\par
            \vspace{5pt}
            {\Large\NameSigPair{\CapstoneSponsorPerson}\par}
            {\large Prepared by }\par
            Group\CapstoneTeamNumber\par
            % 5. comment out the line below this one if you do not wish to name your team
            \CapstoneTeamName\par 
            \vspace{5pt}
            {\Large
                %\NameSigPair{\GroupMemberOne}\par
                %\NameSigPair{\GroupMemberTwo}\par
                \NameSigPair{\GroupMemberThree}\par
                % \NameSigPair{\GroupMemberFour}\par
            }
            \vspace{20pt}
        }
        \begin{abstract}
        % 6. Fill in your abstract
        This document contains a review of three different pieces of technology that will be neccesary to complete the objectives outlined in our requirements document. 
        I will be responsible for reviewing different methods of generating Twitter data, generating Event data, and converting JSON strings into Swift data types. 
        Our project involves the creation of an iOS application by utilizing eBay public buying APIs to help eBay understand the eSports market and provide shopping opportunities. 
        My role is to implement the data generation and display of tweets from targeted eSports accounts in order to make the application more engaging.
            
        	%This document is written using one sentence per line.
        	%This allows you to have sensible diffs when you use \LaTeX with version control, as well as giving a quick visual test to see if sentences are too short/long.
        	%If you have questions, ``The Not So Short Guide to LaTeX'' is a great resource (\url{https://tobi.oetiker.ch/lshort/lshort.pdf})
        \end{abstract}     
    \end{singlespace}
\end{titlepage}
\newpage
\pagenumbering{arabic}
\tableofcontents
% 7. uncomment this (if applicable). Consider adding a page break.
%\listoffigures
%\listoftables
\clearpage

% 8. now you write!
\section{Introduction}
In order to explore the eSports market, eBay would like to test their public buying APIs with a mobile application. 
The core functionality of our app involves displaying eSports merchandise being sold on eBay that is relevant to various eSports games and events. 
However, users will be also able to view a featured eSport event and relevant tweets from targeted accounts on the home screen. 
Part of my role is to implement Twitter functionality in our application so that the most recent tweet from a targeted account is displayed.
In this technical review, I explore different methods of generating Twitter data, converting JSON, and generating event content. 

\section{Piece 1: Generating Twitter Data}
\subsection{Overview}
In order to properly display tweets in our application, we will need to choose a technology for gathering Twitter data.
It is important that the twitter content is related to the featured event being displayed on the home screen.
The goal of this functionality is to provide engaging content and to keep users up to date on current eSports events. 
\subsection{Criteria}
The criteria that will be used to evaluate the different options for gathering Twitter data will be ease of use, dependencies, and documentation.
The options will also be assessed on whether or not the tweets can be dynamically updated and the most recent tweet posted from an account can be displayed. 
\subsection{Potential Choices}
\subsubsection{Twitter Kit for iOS\cite{twitter}}
Twitter Kit for iOS is the official API for integrating Twitter with an iOS application.
The API is free for anyone that has a Twitter account and requires that all requests be authenticated with tokens that can be traced back to an individual Twitter App.
A Twitter App is created in the dashboard and it provides you with an access token that you use to authenticate.
Installing the Twitter Kit can either be done using CocoaPods or manually. 
Once installed, the Twitter Kit allows you to easily display single tweets and timelines from desired accounts in Swift.
Tweet View Style provides UI components for displaying tweets effortlessly. 
\subsubsection{Scraping HTML Data with SwiftSoup\cite{ssoup}}
SwiftSoup is an HTML Parser available under the MIT license which means it is completely free to use. 
It is a pure Swift library that a very convenient API for extracting and manipulating data, using the best of DOM, CSS, and jquery-like methods.
SwiftSoup could be used for generating Twitter data by parsing through HTML that has been scraped from a twitter webpage. 
Implementation would involve providing the URL of a targeted eSports account and then parsing through the HTML for specific tweet data. 
The tweet data would then be stored in a Swift data type that can then be utilized for displaying information in the UI. 
\subsubsection{Manually adding tweets to Firebase\cite{fb}}
Tweets could also be manually stored in Firebase which wouldn't inolve the use of any extra libraries or APIs.
This method wouldn't require integration from any APIs or libaries since we are already using Firebase as our primary database. 
However, entering tweet information manually will be tedious and the content would get old quickly since it wouldn't be updated dynamically.
We would also need to create UI components that display the tweets in a meaningful way which would be time consuming. 
\subsection{Discussion}
The Twitter Kit for iOS is the most well documented option compared to SwiftSoup and using Firebase to manually add tweets.
It would be the easiest to use because it comes packaged with Swift support and UI components which would save a lot of time spent on front-end development. 
The two other options would require us to create our own UI components which would be more difficult. \\ \\
\indent SwiftSoup would require the team to write functions for parsing HTML which would be more complicated than converting JSON with the Twitter Kit or manually entering data into Firebase. 
SwiftSoup has much less documentation that the other two options and would require some testing to make sure Swift 4 is supported. 
Both SwiftSoup and Twitter Kit provide functionality for pulling in the most recent tweet from an account. \\ \\
\indent Manually adding tweets is definitely the least complicated option, but it would require the developer to spend a lot of time updating tweets when new events are featured.
There is plenty of documentation on how to do this and it wouldn't create any additional dependencies since we will already be using Firebase. 
However, manually adding Tweets would not allow us to easily display recent Tweets from a targeted account which is important for creating an engaging experience.
\subsection{Conclusion}
Overall, using the Twitter iOS kit would be the best option for our project because it is well supported by Twitter, has UI components built in, and JSON is easier to convert to Swift code than HTML.

\section{Piece 2: Generating Event Data}
\subsection{Overview}
In order to display events on our home, browse, and events screens, we will need to choose a technology for generating relevant data. 
We plan to curate content and display merchandise for approximately five upcoming eSports events.
Our goal is provide helpful information to the user such as the time and location of these events so they can stay up to date. 
\subsection{Criteria}
The criteria that will be used to evaluate the different options for generating event data will be ease of use, cost, and documentation. 
The options will also be assessed on whether or not there is functionality for pulling in the specific date and location of different events. 
\subsection{Potential Choices}
\subsubsection{Toornament API\cite{toornament}}
Toornament is a free API that gives you access to data of the Toornament platform.
The API sends responses that contain information about different eSports tournaments and games in JSON.
It provides the functionality to list public tournaments, individual matches, participants, and schedules. 
\subsubsection{Abios API\cite{abios}}
Abios is a paid eSports API solution that also sends responses in JSON and covers broad range of data. 
The service includes access to play by plays of many popular games and a betting platform for eSports.
Statistics on team and inter-team performance is also supported. 
Using Abios would require a pricing quote from the company.
\subsubsection{Manually adding events to Firebase\cite{fb}}
Event data could be manually added into Firebase which would utilize JSON strings for each event.
This method wouldn't require integration from any other APIs or libraries since we are already using Firebase as our primary database.
We expect to support approximately five of the largest upcoming global gaming events so the data would be quick to enter and wouldn't change frequently. 
\subsection{Discussion}
During research, I was unable to find an API that targeted a wide range of video game events instead of different game tournaments. 
Toornament is a free option that wouldn't be very difficult to implement, but it only provides score data from different eSports tournaments. 
The documentation is sufficient for what our project would require. \\ \\
\indent Abios provides a huge amount of information compared to Toornament, but much of it would not be used for our application. 
It would be more difficult to use than manually adding data, and would cost more than the other two options which are free.  
Both APIs don't meet our needs because they don't provide the location of different events and there isn't support for large gaming conventions such as E3 and PAX. 
Each of the APIs has a focus on solely tournament information. \\ \\
\indent Using Firebase would make it very easy to generate the content that we would like to display in our application.
All the options would require us to manually add the dates and locations of the largest eSports events that occur each year. 
It is not necessary that our content is generated through APIs because there are only a small number of eSports events we would like to support. 
\subsection{Conclusion}
The main focus of our application is to sell eSports merchandise related to different games and events. 
Displaying tournament information would be useful, but not as important as specific event information that isn't supported by any known APIs.
Due to the ease of manually adding information into Firebase and the small number of eSports events we plan to support, curated content in Firebase looks like the best option for our project.  

\section{Piece 3: Converting JSON}
\subsection{Overview}
We plan to use the eBay Browse API and Twitter API for our application which both return data formatted as JSON strings. 
In order to effectively use the returned data, we will need to convert the JSON into Swift data types and objects that are specific our app’s domain.
Working with JSON can be tedious and there are a number of available technologies that are designed to make it easier.
\subsection{Criteria}
The criteria that will be used to evaluate the different options for converting JSON into Swift data types will be ease of use, support, and documentation. 
\subsection{Potential Choices}
\subsubsection{Apple Foundation Framework\cite{json}}
Apple has implemented a class called JSONSerialization built directly into the Foundation framework that is used to convert JSON into Swift data types.
Swift’s built-in language features make it easy to safely extract and work with JSON data.
\subsubsection{SwiftyJSON Library\cite{sjson}}
The typical JSON handling in Swift is not perfect because it is very strict about explicit types. 
SwiftyJSON is a free JSON parsing library that gives you much clearer syntax when converting JSON in Swift.
It takes out a lot of the headache of parsing through complex JSON data by making use of implicit typing.
\subsubsection{Gloss Library\cite{gloss}}
Gloss is another free JSON parsing library for Swift that has support for mapping JSON to objects, mapping objects to JSON, nested objects, custom transformations.
It involves the creation of compact Gloss models that have roughly 1-line-to-property ratio for deserialization and serialization respectively.
\subsection{Discussion}
The built in JSONSerialization class that is part of the Apple Foundation Framework and is widely used. 
The class is well documented and fully supports Swift 4 since it is part of the default framework. 
It is slightly more difficult to use than the other two options due more complex syntax and more arguments. \\ \\
\indent Compared to Gloss and the Foundation Framework, SwiftyJSON has the simplest syntax when working with JSON since you wouldn't need to create a Gloss model.  
However, SwiftyJSON was recently updated for Swift 4 and may not support all functionality we need for the project. \\ \\ 
\indent Gloss utilizes it's own custom operating syntax to make your model less cluttered, but it makes it difficult to clearly communicate what is happening.
You can disable the custom operators, but then your syntax looks equally as complicated as the built in Swift Foundation Framework, but with more dependencies.

\lstset{language=C++,caption={Apple Foundation Framework Example}}
\begin{lstlisting}
if let statusesArray = try? JSONSerialization.jsonObject(with: data, options: .allowFragments) as? [[String: Any]],
    let user = statusesArray[0]["user"] as? [String: Any],
    let username = user["name"] as? String {
    //Now you have the username
}
\end{lstlisting}

\lstset{language=C++,caption={SwiftyJSON Example}}
\begin{lstlisting}
let json = JSON(data: dataFromNetworking)
if let userName = json[0]["user"]["name"].string {
    //Now you have the username
}
\end{lstlisting}

\lstset{language=C++,caption={Gloss Example (model already implemented)}}
\begin{lstlisting}
Gloss Example
init?(json: JSON) {
    self.username = "name" <~~ json
    //Now you have the username
}
\end{lstlisting}

\subsection{Conclusion}
The default Apple Foundation Framework is widely used in industry and has guaranteed support for Swift 4 and zero dependencies. 
The significant amount of documentation makes it a better choice than Gloss and SwiftyJSON for the purposes of our project. 
Our clients are also most familiar with using the default framework and would like to avoid additional dependencies that would come with the other two options. 
Therefore, we plan to move forward with the JSONSerialization class and Apple Foundation Framework for converting JSON into Swift data types.

\bibliography{tech_review}
\bibliographystyle{IEEEtran}

\end{document}